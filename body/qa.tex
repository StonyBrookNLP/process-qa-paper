% !TEX root =  ../main.tex
\section{Question Answering Using Semantic Roles}

In this section, we describe our approach to answering process recognition questions using the semantic role based representation of processes.
The questions describe a specific instance of a process or a prototypical occurrence of a process. 
The QA task is then to choose the correct process that is being described. 
Intuitively, we can score each process based on how well the roles of the instance described in the question align with the roles of the process we've gathered from other sentences describing the process. 
Higher alignment suggests better evidence. 
In particular, use the following procedure to answer questions:\\
\fbox{
\begin{minipage}{19em}
\begin{enumerate}
\item Convert the multiple-choice question into $k$ statements (${A_1, ..., A_k}$), one for each answer option. 
\item Identify the semantic roles in each question statement by processing them through MATE. 
\item For each statement $A_i$, collect the corresponding process rows from the process knowledge table ($S_1, ..., S_l$). 
\item For each row in the table compute an alignment score by checking for the textual entailment of the corresponding roles. 
\begin{align*}
&alignment(A, S) = \\
& \sum_{role_i \in R} entails(role_i(A), role_i(S))
\end{align*}
where, $R =$\{Input, Result, Enabler, Trigger\}. $entails(x, y)$ is computed as a textual entailment score that reflects how well the text $x$ entails text $y$ or the other way around. It is simply a fraction of word or phrase level coverage of the statement role after accounting for synonyms and hypernyms using WordNet. 
\item Compute the mean of the top 5 alignment scores and use it as the final score for each process. 
\item Return the top scoring process as the answer.
\end{enumerate}
\end{minipage}
}
Note that this scoring procedure doesn't distinguish between definition and instance sentences. 
The textual entailment component is relied upon to provide the type resolution (e.g., water is a type of liquid).



