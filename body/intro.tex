\section{Introduction}

Recent research has shown that a structured approach to knowledge representation, such as through the use of semantic frames, is beneficial to the task of question answering. Resources such as FrameNet and PropBank are invaluable when answering general knowledge type questions, but their lack of coverage makes them insufficient when designing a system to answer domain-specific questions. Unfortunately, domain-specific datasets suitable for this type of problem are basically nonexistent, and manually creating a new dataset of sufficient size is time-consuming and impractical. Semantic role labelling can help to automate this process, but existing SRL systems fail when presented with domain-specific jargon.

In this paper, we present a method for using distant supervision to build a large knowledge base of science concepts represented using semantic frames. Starting with relatively few definition sentences as seeds, we are able to collect many more sentences with which to train a semantic role labeller to perform frame extraction. Finally, we use frame alignment to answer elementary level multiple choice questions in the science domain.

