% !TEX root =  ../main.tex

\subsection{Distant Supervision}
Distant supervision \cite{mintz2009distant},  is an increasingly popular method for relation extraction and typically used in a setting where we have a lot  of unlabeled data and there exist the source of  labeled knowledge base.
They key assumption of distant supervision is " if two entities participate in a relation, any sentence that contain those two entities might express that relation" \cite{mintz2009distant}. 

We make a similar assumption and use distant supervision to obtain more labeled data. We use a the Waterloo corpus~\footnote{This is a large collection of Web documents about 280 GB collected by Charlie Clark at Univ. of Waterloo.} as our unlabeled data source. We search this corpus to find sentences related to each process and automatically annotate them for semantic roles. We use the following procedure:
\begin{itemize}[noitemsep,nolistsep]
\item Collect most frequent role fillers for each processes.  Create a query that finds sentences containing the role fillers.
\item  For each retrieved sentence, identify the location of each role filler. 
Label a candidate role filler only if there is a up-path $\uparrow$ (dependency path) from the candidate role filler nodes to the trigger node. 
If there is no such path then skip the sentence.
\item We also use simple hand crafted  lexical patterns to identify \textit{Input, Enabler and Result}. For example, \textit{Enablers} are often characterized with the preceding words such as \textit{by, through,with, because}. \textit{Results} usually appear after the word \textit{causes, into, produce}. 
\end{itemize}