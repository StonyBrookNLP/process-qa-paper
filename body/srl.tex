% !TEX root =  ../main.tex
\section{Representing Processes via Semantic Roles}

A portion of the grade science exams test the ability to recognize the instances of physical, biological, and other natural processes. 
The questions present a short description of an instance and multiple process names as the answer choices. 

Table below shows a few examples:
\\
\fbox{

\begin{minipage}{18em}

1) As water vapor rises in the atmosphere, it cools and changes back to liquid.
Tiny drops of liquid form clouds in this process called (A) condensation (B) evaporation (C) precipitation (D) run-off.\\
2) The process of change from an egg to an adult butterfly is an example of: (A) vibrations (B) metamorphosis.\\
3) A new aluminum can made from used cans is an example of: (A) recycling (B) reducing (C) reusing (D) repairing	

\end{minipage}
}
\\
\\
The descriptions of the instances are often short. 
They mention the main entities involved and the change brought about by the process. 
At the 4th grade level, the questions do not involve deep knowledge about the sub-events or their sequential order.
Rather the questions test for shallower knowledge about the entities undergoing change, the resulting artifacts, and the main characteristic action describing the process. 
This knowledge is naturally expressed via semantic roles. 
Accordingly we design a simple representation that encodes information about each process via the following roles:
\begin{enumerate}[noitemsep, nolistsep]
\item {\em Input} -- This role captures the main input to the process or the object undergoing the process. e.g., Water is an input to the evaporation process. 
\item {\em Result} -- The artifact that results from the process or the change that results from the process e.g., Water vapor is a result of evaporation. 
\item {\em Trigger} -- The main action, expressed as a verb or its nominalization, indicating the occurrence of the process. e.g., {\em converted} is a trigger for evaporation.
\item {\em Enabler} -- The artifact, condition or action that enables the process to happen. e.g., {\em Sun} is a heat source that is enabler for evaporation.
\end{enumerate}

Our goal is to build aggregate knowledge about processes from multiple sentences. 
If effect we wish to create a table, whose columns are the semantic roles and rows are role fillers obtained from sentences mentioning the process.
We gather knowledge about processes from both definitional sentences and from sentences that describe instances of the processes. 

Definitional sentences present {\em type} information about the roles. 
For instance, the input to evaporation is typically a liquid. 
During question answering, we have to check type compatibility of the roles in the question and the definitional sentence. 
Having instance level information in the process knowledge can help situations where type resolution fails. 
On the other hand, having definitional sentences provides better coverage of information about roles compared to instance sentences,
which can sometimes omit roles that are obvious. 


% !TEX root =  ../main.tex
\subsection{MATE System for SRL}
There have been various  SRL approaches , in terms of resources used (PropBank, FrameNet) and also the techniques and features used. Nevertheless, in general the workflow of an SRL consist of two parts namely argument identification and argument classification.  In this work, we use MATE  SRL system (Bj�rkelund et al., 2009). The reason we use MATE in our setting is because the code is publicly available, it provides automatic predicate/trigger identification, and it is one of the high performing system in  ConLL'09 SRL shared task for English dataset. 

MATE accepts sentences in CoNLL'09 format and processes them through several pipelines namely predicate identification, argument identification, and argument classification. The features used in the pipeline are based on the syntactic information obtained from POS tagger and dependency parser such as the position of the argument with respect to the predicate, the dependency path between the arguments and the predicate, the set of POS tag of the predicate's children etc. Please refer to (Bj�rkelund et al., 2009)f or details.  

The modification that we make to MATE are related to predicate identification and domain adaptation. For the predicate identification, in case of the classifier fails to identify one, we force the classifier to output one predicate based on the sorted confidence score of each of the predicate candidates. For the domain adaptation part, we adopt the simple feature augmentation approach (Hal Daume III, 2007).


% !TEX root =  ../main.tex
Structured prediction problems such as SRL require substantial amounts of training data. 
SRL systems such as MATE typically train on resources such as FrameNet, which contain several \todo{thousand} sentences. 
Semantic roles are not reliably identified with syntactic patterns alone. 
A pattern such as {\em [verb] to [X]} could suggest that X is a result or enabler depending on the process at hand. 
For instance, {\em change to [X]} indicates a resulting state, whereas {\em adapts to [X]} doesn't. 
This suggests that lexical information (e.g., the type of verb) is quite critical. 
Unless the lexical variations had all been observed in the training data, generalization is likely to suffer. 
We explore two ideas to address this issue.

\subsection{Domain Adaptation}

A straightforward approach to training the SRL system is to combine all process sentences into one pool and learn a single model. 
As an alternate approach, we can learn a SRL model for every process using only the sentences that describe the process. 
This enables learning from a much smaller but highly relevant set of sentences. 
Note this is possible in our setting because we know beforehand which process each sentence is describing.
Rather than picking one approach over the other, we can combine their strengths using domain adaptation ideas~\cite{}.


The sentences that describe the target process can be viewed as target domain data, and the sentences describing all other processes can be viewed as the source domain data. Following~\cite{daume} we utilize a simple approach for domain adaptation. The key idea is to take the existing features and create a new feature vector that contains three versions of the original features. A source-specific version, a target-specific version, and a general version. The instances from the source domain will only contain the general and source-specific versions, and the instances for the target domain will contain the general and the target-specific versions. Formally, if $\mathbf{f}$ is the set of features used, then we use the following mappings to create new feature vectors:\\
\begin{align*}
\Phi^{s}(f) &= <f, f, 0>\\
\Phi^{t}(f) &= <f, 0, f>\\
\end{align*}

This mapping enables the learning algorithm to do domain adaptation by learning two sets of weights that reflect the utility of a feature across all domains as well as within the target domain. This simple transformation has been shown to be quite effective for domain adaptation~\cite{}.

\todo{Describe Hal Daume's approach}





% !TEX root =  ../main.tex

\subsection{Distant Supervision}
Distant supervision (Mintz et al , 2009),  is an increasingly popular method for relation extraction and typically used in a setting where we have a lot  of unlabeled data and there exist the source of  labeled knowledge base.
They key assumption of distant supervision is " if two entities participate in a relation, any sentence that contain those two entities might express that relation" (Mintz et al, 2009).  In the context of SRL, the approach from (Furstenau and Lapata, 2012) uses semi supervised technique to find new instances of training data which are similar to the labeled training data by computing structural similarity between the sentences. The formalized the structural similarity  as graph alignment problem.

As in our setting we have very small amount of labeled data and annotation is an expensive effort we try to apply distant supervision to obtain more labeled data. The unlabeled data that we use is the Waterloo corpus. We construct Wumpus query pattern to retrieve the sentences. In particular, we use the following steps:
\begin{itemize}[noitemsep,nolistsep]
\item Collect most frequent role fillers for each processes. 
\item Create the following query pattern that consist of the role fillers.
\item For all the documents retrieved from Wumpus, get the sentence from each document.  For each filtered sentences do the following:
\begin{itemize}[noitemsep,nolistsep]
	\item Matching between token in the sentence and the most frequent role fillers that we have. 
	\item Use the heuristic : only label the candidate role filler if there is a   up-path $\uparrow$ (dependency path) from the candidate role filler nodes to the trigger node. If there is no such path then skip the sentence.
\end{itemize}
\item In addition to the previous steps, we also use some simple hand crafted  lexical patterns  to identify \textit{Undergoer, Enabler and Result}.   For example, \textit{Enablers} are often characterized with the preceding words such as \textit{by, through,with, because}. \textit{Results} usually appear after the word \textit{causes, into, produce}. 
\end{itemize}



