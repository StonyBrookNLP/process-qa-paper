% !TEX root =  ../main.tex
\section{Representing Processes via Semantic Roles}

A portion of the grade science exams test the ability to recognize the instances of physical, biological, and other natural processes. 
The questions present a short description of an instance and multiple process names as the answer choices. 

Table below shows a few examples:
\\
\fbox{

\begin{minipage}{17em}

1) As water vapor rises in the atmosphere, it cools and changes back to liquid.
Tiny drops of liquid form clouds in this process called (A) condensation (B) evaporation (C) precipitation (D) run-off.\\

2) The process of change from an egg to an adult butterfly is an example of: (A) vibrations (B) metamorphosis.\\

3) A new aluminum can made from used cans is an example of: (A) recycling (B) reducing (C) reusing (D) repairing	

\end{minipage}
}
\\
\\
The descriptions of the instances are often short. 
They mention the main entities involved and the change brought about by the process. 
At the 4th grade level, the questions do not involve deep knowledge about the sub-events or their sequential order.
Rather the questions test for shallower knowledge about the entities undergoing change, the resulting artifacts, and the main characteristic action describing the process. 
This knowledge is naturally expressed via semantic roles. 
Accordingly we design a simple representation that encodes information about each process via the following roles:
\begin{enumerate}[noitemsep, nolistsep]
\item {\em Input} -- This role captures the main input to the process or the object undergoing the process. e.g., Water is an input to the evaporation process. 
\item {\em Result} -- The artifact that results from the process or the change that results from the process e.g., Water vapor is a result of evaporation. 
\item {\em Trigger} -- The main action, expressed as a verb or its nominalization, indicating the occurrence of the process. e.g., {\em converted} is a trigger for evaporation.
\item {\em Enabler} -- The artifact, condition or action that enables the process to happen. e.g., {\em Sun} is a heat source that is enabler for evaporation.
\end{enumerate}

Our goal is to build aggregate knowledge about processes from multiple sentences. 
If effect we wish to create a table, whose columns are the semantic roles and rows are role fillers obtained from sentences mentioning the process.
We gather knowledge about processes from both definitional sentences and from sentences that describe instances of the processes. 

Definitional sentences present {\em type} information about the roles. 
For instance, the input to evaporation is typically a liquid. 
During question answering, we have to check type compatibility of the roles in the question and the definitional sentence. 
Having instance level information in the process knowledge can help situations where type resolution fails. 
On the other hand, having definitional sentences provides better coverage of information about roles compared to instance sentences,
which can sometimes omit roles that are obvious. 


% !TEX root =  ../main.tex
\subsection{MATE System for SRL}
% !TEX root =  ../main.tex
Structured prediction problems such as SRL require substantial amounts of training data. 
SRL systems such as MATE typically train on resources such as FrameNet, which contain several \todo{thousand} sentences. 
Semantic roles are not reliably identified with syntactic patterns alone. 
A pattern such as {\em [verb] to [X]} could suggest that X is a result or enabler depending on the process at hand. 
For instance, {\em change to [X]} indicates a resulting state, whereas {\em adapts to [X]} doesn't. 
This suggests that lexical information (e.g., the type of verb) is quite critical. 
Unless the lexical variations had all been observed in the training data, generalization is likely to suffer. 
We explore two ideas to address this issue.

\subsection{Domain Adaptation}

A straightforward approach to training the SRL system is to combine all process sentences into one pool and learn a single model. 
Because we know the process that each sentence is describing, we can learn a SRL model for every process using only the sentences that describe the process. 
This enables learning from a much smaller but highly relevant set of sentences. 
We can combine these approaches using domain adaptation ideas~\cite{}, which allows to combine models from a smaller target domain and larger source domains. 

\todo{Describe Hal Daume's approach}





% !TEX root =  ../main.tex

\subsection{Distant Supervision}
Distant supervision \cite{mintz2009distant},  is an increasingly popular method for relation extraction and typically used in a setting where we have a lot  of unlabeled data and there exist the source of  labeled knowledge base.
They key assumption of distant supervision is " if two entities participate in a relation, any sentence that contain those two entities might express that relation" \cite{mintz2009distant}. 

We make a similar assumption and use distant supervision to obtain more labeled data. We use a the Waterloo corpus~\footnote{This is a large collection of Web documents about 280 GB collected by Charlie Clark at Univ. of Waterloo.} as our unlabeled data source. We search this corpus to find sentences related to each process and automatically annotate them for semantic roles. We use the following procedure:
\begin{itemize}[noitemsep,nolistsep]
\item Collect most frequent role fillers for each processes.  Create a query that finds sentences containing the role fillers.
\item  For each retrieved sentence, identify the location of each role filler. 
Label a candidate role filler only if there is a up-path $\uparrow$ (dependency path) from the candidate role filler nodes to the trigger node. 
If there is no such path then skip the sentence.
\item We also use simple hand crafted  lexical patterns to identify \textit{Input, Enabler and Result}. For example, \textit{Enablers} are often characterized with the preceding words such as \textit{by, through,with, because}. \textit{Results} usually appear after the word \textit{causes, into, produce}. 
\end{itemize}



